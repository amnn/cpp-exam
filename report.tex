\documentclass{tufte-handout}

\title{Categories, Proofs and Processes 2015}
\author{Candidate No. 683444}

\usepackage{amsmath}
\usepackage{amssymb}
\usepackage{amsthm}
\usepackage{bm}
\usepackage{bussproofs}
\usepackage{calc}
\usepackage{enumerate}
\usepackage{mathtools}
\usepackage{relsize}
\usepackage{stmaryrd}
\usepackage{tikz}
\usepackage{wasysym}

\usetikzlibrary{cd}

%%% Title Formatting

\titleformat*{\section}{\normalfont\Large\scshape}
\titleformat*{\subsection}{\normalfont\large\scshape}

%%% Custom Commands

\newcommand{\Lemma}{\textbf{lemma}}
\newcommand{\Thm}{\textbf{thm}}
\newcommand{\Def}{\textbf{def}}
\newcommand{\Contra}{\textbf{contra}}
\newcommand{\Assoc}{\textbf{assoc}}
\newcommand{\Trans}{\textbf{trans}}
\newcommand{\Refl}{\textbf{refl}}
\newcommand{\Hyp}{\textbf{hyp}}
\newcommand{\Ass}{\textbf{assume}}
\newcommand{\Intro}{\textbf{intro}}
\newcommand{\Elim}{\textbf{elim}}
\newcommand{\Let}{\textbf{let}}
\newcommand{\Epic}{\textbf{epic}}
\newcommand{\Monic}{\textbf{monic}}
\newcommand{\compose}{\circ}
\newcommand{\epic}{\twoheadrightarrow}
\newcommand{\cat}[1]{\mathcal{#1}}
\newcommand{\opcat}[1]{\mathcal{#1}^{\text{op}}}
\renewcommand{\hom}[3]{\cat{#1}(#2, #3)}
\newcommand{\Exp}{\Rightarrow}
\newcommand{\sembrack}[1]{\llbracket #1 \rrbracket}

\newcommand{\step}[1][\phantom{=}]{\item[{\makebox[{\widthof{$\Leftrightarrow$}}][l]{$#1$}}]}
\newcommand{\subp}[1]{\item[$#1$]}
\newcommand{\iffs}{\Leftrightarrow}
\newcommand{\imps}{\Rightarrow}
\newcommand{\contras}{\divideontimes}

\def\mathnote#1{%
  \tag*{\rlap{\hspace\marginparsep\smash{\parbox[t]{\marginparwidth}{%
  \footnotesize#1}}}}
}

\DeclarePairedDelimiter\abs{\lvert}{\rvert}
\DeclareMathOperator{\id}{id}
\DeclareMathOperator{\FV}{FV}
\DeclareMathOperator{\App}{App}

%%% Theorem styles

\theoremstyle{definition}
\newtheorem{definition}{Definition}
\numberwithin{definition}{section}

\theoremstyle{plain}
\newtheorem{prop}{Proposition}
\numberwithin{prop}{section}

\theoremstyle{plain}
\newtheorem{lemma}{Lemma}
\numberwithin{lemma}{section}

%%% Initialise Counters

\setcounter{section}{1}

%%% Proof Trees
\EnableBpAbbreviations

%%% Content
\begin{document}
\maketitle

\section{Question 1}\label{sec:q-1}
\subsection{Part (a.i)}\label{sec:q-1-a-i}
\subsection{Part (a.ii)}\label{sec:q-1-a-ii}
\subsection{Part (a.iii)}\label{sec:q-1-a-iii}
\subsection{Part (b.i)}\label{sec:q-1-b-i}
\subsection{Part (b.ii)}\label{sec:q-1-b-ii}
\subsection{Part (b.iii)}\label{sec:q-1-b-iii}
\subsection{Part (b.iv)}\label{sec:q-1-b-iv}
\subsection{Part (b.v)}\label{sec:q-1-b-v}
\subsection{Part (b.vi)}\label{sec:q-1-b-vi}
\subsection{Part (b.vii)}\label{sec:q-1-b-vii}


\section{Question 2}\label{sec:q-2}
\subsection{Part (a)}\label{sec:q-2-a}
\subsection{Part (b)}\label{sec:q-2-b}
\subsection{Part (c)}\label{sec:q-2-c}
\subsection{Part (d)}\label{sec:q-2-d}
\subsection{Part (e)}\label{sec:q-2-e}
\subsection{Part (f)}\label{sec:q-2-f}
\subsection{Part (g)}\label{sec:q-2-g}
\subsection{Part (h)}\label{sec:q-2-h}
\subsection{Part (i)}\label{sec:q-2-i}
\subsection{Part (j)}\label{sec:q-2-j}
\subsection{Part (k)}\label{sec:q-2-k}
\subsection{Part (l)}\label{sec:q-2-l}


\end{document}
