\subsection{Part (a.i)}\label{sec:q-1-a-i}

First, we must define a notation for the objects in $\mathbf{Set}$ that our functors will map objects to. The image of both functors is restricted to sets of this form.

\begin{definition}[Hom-set]\label{def:hom-set}
  $\hom{C}{A}{B}\in\mathbf{Set}$. The set of all morphisms from object $A$ to object $B$ in category $\cat{C}$.
\end{definition}

Then we define the action of the functor on objects and morphisms. The covariant Hom-functor maps objects to the set of morphisms \textit{to} that object from $A$, and maps morphisms to functions between Hom-sets (given in the definition).

\begin{definition}[Covariant Hom-functor]\label{def:co-hom}
  $\hom{C}{A}{-} : \cat{C} \to \mathbf{Set}$
  \marginnote{Let $\cat{C}$ be a category, $A\in\cat{C}$}
  \begin{align*}
    \hom{C}{A}{-} & : B \mapsto \hom{C}{A}{B}\\
    \hom{C}{A}{-} & : (f : B \to C) \mapsto (g \mapsto f \compose g)
  \end{align*}
\end{definition}

\begin{marginfigure}[10em]
  \begin{tikzcd}[sep=large]
    & \hom{C}{A}{D} \\
      \hom{C}{A}{B} \arrow[ru,"\hom{C}{A}{g\compose h}"]
                    \arrow[r, "\hom{C}{A}{h}"']
    & \hom{C}{A}{C} \arrow[u, "\hom{C}{A}{g}"']
  \end{tikzcd}
  \caption{Preservation of composition in the image of $\hom{C}{A}{-}$.}\label{fig:co-functor}
\end{marginfigure}

\begin{prop}
  $\hom{C}{A}{-}$ is a functor.

  It suffices to prove that:
  \begin{enumerate}[(i)]
    \item $\hom{C}{A}{-}$ preserves identity morphisms. That is to say, for any $B\in\cat{C}$ it follows that $\hom{C}{A}{\id_B} = \id_{\hom{C}{A}{B}}$
    \item $\hom{C}{A}{-}$ preserves composition. That is to say, for any $B,C,D\in\cat{C}$ with morphisms $g : C \to D$
      and $h : B \to C$, Figure~\ref{fig:co-functor} commutes.
  \end{enumerate}
  \begin{proof}[Proof (i)]
    Let $B\in\cat{C}$
    \begin{itemize}
      \step
        \begin{itemize}
          \subp{\star}
            \Let~$g : A \to B$
            \marginnote{\Hyp}
          \step
            $\hom{C}{A}{\id_B}(g)$

          \step[=]
            $\id_B\compose g$
            \marginnote{\Def-\ref{def:co-hom}}

          \step[=]
            $g$
            \marginnote{\Def-$\id$}

          \step[=]
            $\id_{\hom{C}{A}{B}}(g)$
            \marginnote{\Def-$\id$}
        \end{itemize}

      \step[\imps]
        $\forall g : A \to B\ldotp~\hom{C}{A}{B}(g) = \id_{\hom{C}{A}{B}}(g)$
        \marginnote{$\forall$-\Intro}

      \step[\iffs]
        $\hom{C}{A}{B} = \id_{\hom{C}{A}{B}}$
        \marginnote{\Def-=}
        \qedhere
    \end{itemize}
  \end{proof}

  \begin{proof}[Proof (ii)]
    Let $B, C, D\in\cat{C}$ with ${g : C\to D}$, ${h : B\to C}$.
    \begin{itemize}
      \step
        \begin{itemize}
          \subp{\star} \Let~$i : A\to B$
            \marginnote{\Hyp}

          \step
            $(\hom{C}{A}{g}\compose\hom{C}{A}{h})(i)$

          \step[=] $\hom{C}{A}{g}(\hom{C}{A}{h}(i))$
            \marginnote{\Def-$\compose$}

          \step[=] $\hom{C}{A}{g}(h\compose i)$
            \marginnote{\Def-\ref{def:co-hom}}

          \step[=] $g\compose (h\compose i)$
            \marginnote{\Def-\ref{def:co-hom}}

          \step[=] $(g\compose h)\compose i$
            \marginnote{\Assoc-$\compose$}

          \step[=] $\hom{C}{A}{g\compose h}(i)$
            \marginnote{\Def-\ref{def:co-hom}}
        \end{itemize}

      \step[\imps]
        $\forall i : A\to B\ldotp~(\hom{C}{A}{g}\compose\hom{C}{A}{h})(i) = \hom{C}{A}{g\compose h}(i)$
        \marginnote{$\forall$-\Intro}

      \step[\iffs]
        $\hom{C}{A}{g}\compose\hom{C}{A}{h} = \hom{C}{A}{g\compose h}$
        \marginnote{\Def-$=$}
        \qedhere
    \end{itemize}
  \end{proof}
\end{prop}

Similarly the contravariant Hom-functor maps objects to the set of morphisms \textit{from} that object to A, and maps morphisms to functions between Hom-sets in a corresponding (but not equivalent) way (see below).

\begin{definition}[Contravariant Hom-functor]\label{def:contra-hom}
  $\hom{C}{-}{A} : \opcat{C} \to \mathbf{Set}$
  \marginnote{Let $\cat{C}$ be a category, $A\in\cat{C}$}
  \begin{align*}
    \hom{C}{-}{A} & : B \mapsto \hom{C}{B}{A}\\
    \hom{C}{-}{A} & : (f : B \to C) \mapsto (g \mapsto g \compose f)
  \end{align*}
\end{definition}

\begin{marginfigure}[6em]
  \begin{tikzcd}[sep=large]
    & \hom{C}{D}{A} \\
      \hom{C}{B}{A} \arrow[ru,"\hom{C}{h\compose g}{A}"]
                    \arrow[r, "\hom{C}{h}{A}"']
    & \hom{C}{C}{A} \arrow[u, "\hom{C}{g}{A}"']
  \end{tikzcd}
  \caption{Preservation of composition in the image of $\hom{C}{-}{A}$. Note that the composition $h\compose g$ has been reversed by the definition of composition in the opposite category.}\label{fig:contra-functor}
\end{marginfigure}

\begin{prop}
  $\hom{C}{-}{A}$ is a functor.

  It suffices to prove that:
  \begin{enumerate}[(i)]
    \item $\hom{C}{-}{A}$ preserves identity morphisms. That is to say, for any $B\in\cat{C}$ it follows that $\hom{C}{\id_B}{A} = \id_{\hom{C}{B}{A}}$
    \item $\hom{C}{-}{A}$ preserves composition. That is to say, for any $B,C,D\in\cat{C}$ with morphisms $g : C \to D$
      and $h : B \to C$, Figure~\ref{fig:contra-functor} commutes.
  \end{enumerate}

  \begin{proof}[Proof (i)]
    Let $B\in\cat{C}$
    \begin{itemize}
      \step
        \begin{itemize}
          \subp{\star}
            \Let~$g : B \to A$
            \marginnote{\Hyp}
          \step
            $\hom{C}{\id_B}{A}(g)$

          \step[=]
            $g\compose\id_B$
            \marginnote{\Def-\ref{def:contra-hom}}

          \step[=]
            $g$
            \marginnote{\Def-$\id$}

          \step[=]
            $\id_{\hom{C}{B}{A}}(g)$
            \marginnote{\Def-$\id$}
        \end{itemize}

      \step[\imps]
        $\forall g : B \to A\ldotp~\hom{C}{\id_B}{A}(g) = \id_{\hom{C}{B}{A}}(g)$
        \marginnote{$\forall$-\Intro}

      \step[\iffs]
        $\hom{C}{B}{A} = \id_{\hom{C}{B}{A}}$
        \marginnote{\Def-=}
        \qedhere
    \end{itemize}
  \end{proof}

  \begin{proof}[Proof (ii)]
    Let $B, C, D\in\cat{C}$ with ${g : C\to D}$, ${h : B\to C}$.
    \begin{itemize}
      \step
        \begin{itemize}
          \subp{\star} \Let~$i : B\to A$
            \marginnote{\Hyp}

          \step
            $(\hom{C}{g}{A}\compose\hom{C}{h}{A})(i)$

          \step[=] $\hom{C}{g}{A}(\hom{C}{h}{A}(i))$
            \marginnote{\Def-$\compose$}

          \step[=] $\hom{C}{g}{A}(i\compose h)$
            \marginnote{\Def-\ref{def:contra-hom}}

          \step[=] $(i\compose h)\compose g$
            \marginnote{\Def-\ref{def:contra-hom}}

          \step[=] $i\compose (h\compose g)$
            \marginnote{\Assoc-$\compose$}

          \step[=] $\hom{C}{h\compose g}{A}(i)$
            \marginnote{\Def-\ref{def:contra-hom}}
        \end{itemize}

      \step[\imps]
        $\forall i : B\to A\ldotp~(\hom{C}{g}{A}\compose\hom{C}{h}{A})(i) = \hom{C}{h\compose g}{A}(i)$
        \marginnote{$\forall$-\Intro}

      \step[\iffs]
        $\hom{C}{g}{A}\compose\hom{C}{h}{A} = \hom{C}{h\compose g}{A}$
        \marginnote{\Def-$=$}
        \qedhere
    \end{itemize}
  \end{proof}
\end{prop}

\subsection{Part (a.ii)}\label{sec:q-1-a-ii}

\begin{prop}
  $f$ epic $\iff$ $\hom{C}{f}{C}$ injective for any $C\in\cat{C}$
  \marginnote{Let $f : A \to B$ be a morphism in category $\cat{C}$.}

  \begin{proof}[Proof $\Rightarrow$]
    Let $C\in\cat{C}$.
    \begin{itemize}
      \step
        \begin{itemize}
          \subp{\star} \Let~$g, h : B\to C$
            \marginnote{\Hyp}

          \step
            \begin{itemize}
              \subp{\dagger}
                \Ass~$\hom{C}{f}{C}(g) = \hom{C}{f}{C}(h)$
                \marginnote{\Hyp}

            \step[\iffs] $g\compose f = h\compose f$
              \marginnote{\Def-\ref{def:contra-hom}}

            \step[\imps] $g = h$
              \marginnote{\Hyp-($f$~\Epic)}
            \end{itemize}

          \step[\imps] $\hom{C}{f}{C}(g) = \hom{C}{f}{C}(h) \implies g = h$
            \marginnote{$\imps$-\Intro}
        \end{itemize}
      \step[\imps] $\forall g, h : B \to C\ldotp~\hom{C}{f}{C}(g) = \hom{C}{f}{C}(h) \implies g = h$
        \marginnote{$\forall$-\Intro}

      \step[\iffs] $\hom{C}{f}{C}$ injective
        \marginnote{\Def-injective}
        \qedhere
    \end{itemize}
  \end{proof}

  \begin{proof}[Proof $\Leftarrow$]
    Let $C\in\cat{C}$.
    \begin{itemize}
      \step
        \begin{itemize}
          \subp{\star}
            \Let~$g, h : B\to C$
            \marginnote{\Hyp}

          \step
            \begin{itemize}
            \subp{\dagger}
              \Ass~$g\compose f = h\compose f$
              \marginnote{\Hyp}

            \step[\iffs] $\hom{C}{f}{C}(g) = \hom{C}{f}{C}(h)$
              \marginnote{\Def-\ref{def:contra-hom}}

            \step[\imps] $g = h$
              \marginnote{\Hyp-($\hom{C}{f}{C}$~\textbf{injective})}
            \end{itemize}

          \step[\imps] $g\compose f = h\compose f \implies g = h$
            \marginnote{$\imps$-\Intro}
        \end{itemize}
      \step[\imps] $\forall g, h : B \to C\ldotp~g\compose f = h\compose f \implies g = h$
        \marginnote{$\forall$-\Intro}

      \step[\iffs] $f$ epic
        \marginnote{\Def-epic}
        \qedhere
    \end{itemize}
  \end{proof}
\end{prop}

\subsection{Part (a.iii)}\label{sec:q-1-a-iii}
By dualising the result in part (a.ii), we get a characterisation of monics as follows: $f : B\to C$ in $\cat{C}$ is monic if and only if $\hom{C}{A}{f}$ is injective for any $A\in\cat{C}$. The difference being that now the characterisation uses the \textit{covariant} Hom-functor where earlier, when describing epics, it was in terms of the \textit{contravariant} Hom-functor.

\begin{prop}
  $f$ monic $\iff$ $\hom{C}{A}{f}$ injective for any $A\in\cat{C}$
  \marginnote{Let $f : B \to C$ be a morphism in category $\cat{C}$.}

  \begin{proof}[Proof $\Rightarrow$]
    Let $A\in\cat{C}$.
    \begin{itemize}
      \step
        \begin{itemize}
          \subp{\star} \Let~$g, h : A\to B$
            \marginnote{\Hyp}

          \step
            \begin{itemize}
            \subp{\dagger}
              \Ass~$\hom{C}{A}{f}(g) = \hom{C}{A}{f}(h)$
              \marginnote{\Hyp}

            \step[\iffs] $f\compose g = f\compose h$
              \marginnote{\Def-\ref{def:co-hom}}

            \step[\imps] $g = h$
              \marginnote{\Hyp-($f$~\Monic)}
            \end{itemize}

          \step[\imps] $\hom{C}{A}{f}(g) = \hom{C}{A}{f}(h) \implies g = h$
            \marginnote{$\imps$-\Intro}
        \end{itemize}
      \step[\imps] $\forall g, h : A \to B\ldotp~\hom{C}{A}{f}(g) = \hom{C}{A}{f}(h) \implies g = h$
        \marginnote{$\forall$-\Intro}

      \step[\iffs] $\hom{C}{A}{f}$ injective
        \marginnote{\Def-injective}
        \qedhere
    \end{itemize}
  \end{proof}

  \begin{proof}[Proof $\Leftarrow$]
    Let $A\in\cat{C}$.
    \begin{itemize}
      \step
        \begin{itemize}
          \subp{\star} \Let~$g, h : A\to B$
            \marginnote{\Hyp}

          \step
            \begin{itemize}
            \subp{\dagger}
              \Ass~$f\compose g = f\compose h$
              \marginnote{\Hyp}

            \step[\iffs] $\hom{C}{A}{f}(g) = \hom{C}{A}{f}(h)$
              \marginnote{\Def-\ref{def:co-hom}}

            \step[\imps] $g = h$
              \marginnote{\Hyp-($\hom{C}{A}{f}$~\textbf{injective})}
            \end{itemize}

          \step[\imps] $f\compose g = f\compose h \implies g = h$
            \marginnote{$\imps$-\Intro}
        \end{itemize}
      \step[\imps] $\forall g, h : A \to B\ldotp~f\compose g = f\compose h \implies g = h$
        \marginnote{$\forall$-\Intro}

      \step[\iffs] $f$ monic
        \marginnote{\Def-monic}
        \qedhere
    \end{itemize}
  \end{proof}
\end{prop}

\subsection{Part (b.i)}\label{sec:q-1-b-i}

In order to prove the main result in this question, it is useful to construct a lemma:

\begin{lemma}\label{lemma:epic-preserve-1}
  For any epic $e : B \epic C$:
  \marginnote{Let $\cat{C}$ be a category and $E\in\cat{C}$.}

  $(\forall f : E\to C\ldotp\exists h : E\to B\ldotp~e\compose h = f)$
  $\iff$ $\hom{C}{E}{e}$ is epic.

  \begin{proof}
    Consider epic $e : B \epic C$ in $\cat{C}$.
    \begin{itemize}
      \step $\forall f : E\to C\ldotp\exists h : E\to B\ldotp~e\compose h = f$
        \marginnote{\Hyp}
      \step[\iffs]

    $\forall f \in\hom{C}{E}{C}\ldotp\exists h\in\hom{C}{E}{B}\ldotp~\hom{C}{E}{e}(h) = f$
        \marginnote{\Def~\ref{def:hom-set}}

      \step[\iffs]
        $\hom{C}{E}{e}$ surjective.
        \marginnote{\Def-surjective}

      \step[\iffs]
        $\hom{C}{E}{e}$ epic
        \qedhere
        \marginnote{\textbf{sheet} 1, \textbf{q}1}

    \end{itemize}
  \end{proof}
\end{lemma}

Our intended result then follows from the previous lemma, and a general result from first-order logic:

\begin{lemma}\label{lemma:epic-preserve-2}
  If, given sets $X, P, Q$, for any $x\in X$ we have that ${P(x) \iff Q(x)}$ then $\forall x\in X\ldotp~P(x)$ iff $\forall x\in X\ldotp~Q(x)$.

  \begin{proof}
    Let $X, P, Q$ be sets
    \begin{itemize}
      \step
        \begin{itemize}
          \subp{\star}
            \Ass~$\forall x\in X\ldotp~P(x)^{\text~(1)}$
            \marginnote{\Hyp}

          \step
            \begin{itemize}
              \subp{\dagger}
                \Let~$x\in X$
                \marginnote{\Hyp}

              \step
                $P(x)$
                \marginnote{$\forall$-\Elim-(1)}

              \step
                $P(x)\iff Q(x)$
                \marginnote{$\forall$-\Elim-($\forall x\in X\ldotp~P(x)\iff Q(x)$)}

              \step[\imps]
                $Q(x)$
                \marginnote{$\iffs$-\Elim}
            \end{itemize}

            \step[\imps]
              $\forall x\in X\ldotp~Q(x)$
              \marginnote{$\forall$-\Intro}
        \end{itemize}

      \step[\imps]
        $(\forall x\in X\ldotp~P(x))\implies(\forall x\in X\ldotp~Q(x))$
        \marginnote{$\imps$-\Intro}
      \step[\phantom{\imps}]
        $(\forall x\in X\ldotp~Q(x))\implies(\forall x\in X\ldotp~P(x))$
        \marginnote{follows from a symmetric argument}
      \step[\iffs]
        $(\forall x\in X\ldotp~P(x))\iff(\forall x\in X\ldotp~Q(x))$
        \qedhere
        \marginnote{$\iffs$-\Intro}
    \end{itemize}
  \end{proof}
\end{lemma}

All that remains to be done, is to join the two lemmas together (the conclusion of Lemma~\ref{lemma:epic-preserve-1} satisfies the antecedent of Lemma~\ref{lemma:epic-preserve-2}, for a given $X$, $P$, and $Q$):

\begin{marginfigure}
  \begin{tikzcd}[sep=large]
    & B \arrow[d, two heads, "e"]\\
      E \arrow[ur, dashed, "h"]
        \arrow[r, "f"']
    & C
  \end{tikzcd}
  \caption{Epic preserving object, $E$.}\label{fig:epic-preserving}
\end{marginfigure}

\begin{prop}\label{prop:epic-preserving}
  $E$ preserves epics $\iff$ for every epic $e : B \epic C$, and every morphism $f : E \to C$, there exists $h : E \to B$ such that Figure~\ref{fig:epic-preserving} commutes.

  \begin{proof}
    Let $E\in\cat{C}$.
    \begin{itemize}
      \step
        $E$ preserves epics
        \marginnote{\Hyp}

      \step[\iffs]
        $\hom{C}{E}{-}$ preserves epics
        \marginnote{\Def-(preserves epics)}

      \step[\iffs]
        $\forall e : B \epic C\ldotp$ $\hom{C}{E}{e}$ is epic.
        \marginnote{\Def-(preserves epics)}

      \step[\iffs]
        $\forall e : B\epic C\ldotp~\forall f : E \to C\ldotp\exists h : E \to B\ldotp e\compose h = f$
        \qedhere
        \marginnote{\Lemma-\ref{lemma:epic-preserve-2}-(\Lemma-\ref{lemma:epic-preserve-1})}
    \end{itemize}
  \end{proof}
\end{prop}

\subsection{Part (b.ii)}\label{sec:q-1-b-ii}

\begin{prop}
  If $E$ preserves epics then any retract $F$ of $E$ also preserves epics.

  \begin{proof}
    Suppose $E\in\cat{C}$ preserves epics$.^{\text~(1)}$

    \begin{itemize}
      \step
        \begin{itemize}
          \subp{\star}
            \Let~$F \in \cat{C}$, $r : E \rightleftarrows F : s$ be a retract of $E^{\text~(2)}$
            \marginnote{\Hyp}

          \step
            \begin{itemize}
              \subp{\dagger}
                \Let~$e : B \epic C^{\text~(3)}$
                \marginnote{\Hyp}

              \step
                \Let~$f : F \to C$
                \marginnote{\Hyp}

              \step[\imps]
                The following diagram commutes
                \marginnote[2em]{The top half of this diagram commutes because of assumptions (1) and (3) above, whilst the bottom half commutes because of assumption (2).}
                \begin{center}
                  \begin{tikzcd}[sep=large]
                    && B \arrow[d, two heads, "e"]
                    \\ E \arrow[r, "r"'] \arrow[urr, dashed, "h"]
                    &  F \arrow[r, "f"']
                    &  C
                    \\ F \arrow[u, "s"] \arrow[ur, "\id_F"']
                  \end{tikzcd}
                \end{center}

              \step[\imps] $f\compose\id_F = e\compose h\compose s$
                \marginnote{\textbf diagram chase}

              \step[\imps] $f = e\compose h\compose s$
                \marginnote{\Def-$\id$}

              \step[\imps] $\exists h^\prime : F\to B\ldotp~f = e\compose h^\prime$
                \marginnote{$\exists$-\Intro}
            \end{itemize}
            \step[\imps]
              $\forall e : B \epic C\ldotp~\forall f : F \to C\ldotp~\exists h^\prime : F\to B\ldotp~f = e\compose h^\prime$
              \marginnote{$\forall$-\Intro}

            \step[\iffs] $F$ preserves epics
              \marginnote{\Thm-\ref{prop:epic-preserving}}
              \qedhere
        \end{itemize}
    \end{itemize}
  \end{proof}
\end{prop}

\subsection{Part (b.iii)}\label{sec:q-1-b-iii}

\begin{prop}\label{prop:set-epic-preserving}
  For any $X\in\mathbf{Set}$, $X$ is epic-preserving.

  \begin{proof}
    Let $X\in\mathbf{Set}$.
    \begin{itemize}
      \step
        \begin{itemize}
          \subp{\star}
            \Let~$e : B\epic C$
            \marginnote{\Hyp; $B, C\in\mathbf{Set}$}

          \step\Let~$f : X\to C$
            \marginnote{\Hyp}

            \step $e$ epic in $\mathbf{Set}$

            \step[\iffs] $e$ surjective
              \marginnote{\textbf{sheet} 1 \textbf{q}1}

            \addtolength{\itemsep}{.5\baselineskip}
            \step
              Consider the set family $\{B_c\}_{c\in C}$\\
              where $B_c \triangleq \{b\in B : e(b) = c\}$

            \step
              $\{B_c\}_{c\in C}$ is a family of non-empty sets
              \marginnote {$e$ \textbf{surjective}}

            \addtolength{\itemsep}{-.5\baselineskip}
            \step[\imps]
              We may construct a set $\{b_c\}_{c\in C}$ such that $\forall c\in C\ldotp~b_c\in B_c$.
              \marginnote{Axiom of Choice}\\[1\baselineskip]
              In general $\{B_c\}_{c\in C}$ may be an infinite family. In such cases, there is no terminating procedure by which we may create $\{b_c\}_{c\in C}$, and we have no choice function to witness this set either, so we cannot reliably assert its existence without the Axiom of Choice.

            \addtolength{\itemsep}{+1\baselineskip}
            \step
              \Let~$h : X \to B;~h : x \mapsto b_{f(x)}$
            \addtolength{\itemsep}{-1\baselineskip}

          \step
              \begin{itemize}
                \subp{\dagger}
                  \Let~$x\in X$
                  \marginnote{\Hyp}

                \step $(e\compose h)(x)$

                \step[=] $e(b_{f(x)})$
                  \marginnote{\Def-$h$}

                \step[=] $f(x)$
                  \marginnote{$b_{f(x)}\in B_{f(x)}$}
              \end{itemize}

              \step[\imps]
                $\forall x\in X\ldotp~(e\compose h)(x) = f(x)$
                \marginnote{$\forall$-\Intro}

              \step[\iffs]
                $e\compose h = f$
                \marginnote{\Def-$=$}

              \step[\imps]
                $\exists h : X \to B\ldotp~e\compose h = f$
                \marginnote{$\exists$-\Intro}
        \end{itemize}

        \step[\imps]
          $\forall e : B\epic C\ldotp~\forall f : X \to C\ldotp~\exists h : X \to B\ldotp~e\compose h = f$
          \marginnote{$\forall$-\Intro}

        \step[\iffs]
          $X$ preserves epics
          \qedhere
          \marginnote{\Thm-\ref{prop:epic-preserving}}
    \end{itemize}
  \end{proof}
\end{prop}

\subsection{Part (b.iv)}\label{sec:q-1-b-iv}

Our proof of this result hinges on two facts, which we will present as lemmas. Firstly, that any epic morphism in $\mathbf{Pos}$ is also epic (that is to say, surjective) in $\mathbf{Set}$. And secondly, if a poset is discrete, any map from it is necessarily monotone.

\begin{lemma}\label{lemma:epic-poset-set}
  If $e : (P, \preceq_P) \epic (Q, \preceq_Q)$ in $\mathbf{Pos}$ then $e : P \epic Q$ in $\mathbf{Set}$.

  \begin{proof}[Proof by Contradiction]
    Suppose for a contradiction that $e$ is not epic in $\mathbf{Set}$.

    \begin{itemize}
      \step[\iffs] $e$ is not surjective
        \marginnote{\textbf{sheet} 1 \textbf{q}1}

      \step[\imps] There exists some $\widehat{q}\in Q\setminus e(P)$.
        \marginnote{\Def-$\neg$surjective}

      \addtolength{\itemsep}{.5\baselineskip}
      \step
        Define $g,~h : (Q, \preceq_Q) \to (\mathbb{B}, \preceq_{\mathbb{B}})$\\
        where $\mathbb{B}\triangleq\{0 \preceq_{\mathbb{B}} 1\}$

        \begin{align*}
          g & : q \mapsto
          \begin{cases}
            1 & \text{if }\widehat{q}\preceq_Q q\\
            0 & \text{otherwise}
          \end{cases}\\
          h & : q \mapsto
          \begin{cases}
            1 & \text{if }\widehat{q}\preceq_Q q\wedge\widehat{q}\neq q\\
            0 & \text{otherwise}
          \end{cases}
        \end{align*}

        We now show that both $g$ and $h$ are monotone, by a proof by contradiction:

      \addtolength{\itemsep}{-.5\baselineskip}
      \step
        \begin{itemize}
          \subp{\star}
            \Ass~$g$ not monotone
            \marginnote{\Hyp}

          \step[\imps]
            $\exists a, b\in Q\ldotp~a\preceq_Qb\wedge g(a)\npreceq_{\mathbb{B}}g(b)$
            \marginnote{\Def-$\neg$monotone}

          \step[\imps]
            $a\preceq_Qb\land g(a)\succeq_{\mathbb{B}}g(b)\land g(a)\neq g(b)$
            \marginnote{$\preceq_{\mathbb{B}}$~\textbf{total}}

          \step[\imps]
            $a\preceq_Qb\wedge g(a) = 1\wedge g(b) = 0$
            \marginnote{\Def-$\mathbb{B}$}

          \step[\imps]
            $a\preceq_Qb\wedge \widehat{q}\preceq_Qa\wedge \widehat{q}\npreceq_Qb$
            \marginnote{\Def-$g$}

          \step[\imps]
            $\widehat{q}\preceq_Qb\wedge \widehat{q}\npreceq_Qb$
            \marginnote{\Trans-$\preceq_Q$}
        \end{itemize}

      \step[\contras] $g$ monotone
        \marginnote{\Contra}

      \step $h$ monotone
        \marginnote{similar argument to $\star$.}

      \addtolength{\itemsep}{.5\baselineskip}
      \step Given that we now know $g$ and $h$ are morphisms in $\mathbf{Pos}$, we can use the \Epic-ness of $e$ to derive a contradiction of our original assumption:

      \step
        \begin{itemize}
          \subp{\dagger}
            $\forall q\in Q\setminus\{\widehat{q}\}\ldotp~g(q) = h(q)$
            \marginnote{\Def-$g$;~\Def-$h$}

          \step[\imps]
            $\forall p\in P\ldotp~g(e(p)) = h(e(q))$
            \marginnote{$e(P)\subseteq Q\setminus\{\widehat{q}\}$}


          \step[\iffs]
            $g\compose e = h\compose e$
            \marginnote{\Def-$=$}

          \step[\imps]
            $g = h$
            \marginnote{$e$ \Epic}
        \end{itemize}
      \addtolength{\itemsep}{-.5\baselineskip}

      \step
        \begin{itemize}
          \subp{\ddagger}
            $g(\widehat{q})\neq h(\widehat{q})$
            \marginnote{\Def-$g$;~\Def-$h$}

          \step[\imps]
            $g\neq h$
            \marginnote{\Def-$\neq$}
        \end{itemize}

        \step[\contras] $e$ epic in $\mathbf{Set}$
          \qedhere
          \marginnote{\Contra-$\dagger$}
    \end{itemize}
  \end{proof}
\end{lemma}

\begin{lemma}\label{lemma:discrete-poset-monotone}
  For any $(S, \preceq_S)\in\mathbf{Pos}$ and any discrete $(P, \preceq_P)\in\mathbf{Pos}$, every function $g : P \to S$ is trivially monotone.

  \begin{proof}
    Let $x, y\in P$, such that $x\preceq_Py$
    \begin{itemize}
      \step[\imps] $x = y$
        \marginnote {$P$ \textbf{discrete}}

      \step[\imps] $g(x) = g(y)$

      \step[\imps] $g(x) \preceq_S g(y)$
        \qedhere
        \marginnote{\Refl-$S$}
    \end{itemize}

  \end{proof}
\end{lemma}

\begin{prop}
  Every discrete poset is epic-preserving in $\mathbf{Pos}$.

  \begin{proof}
    Let $(P, \preceq_P)$ be a \textit{discrete} poset.
    \begin{itemize}
      \step
        \begin{itemize}
          \subp{\star}
            \Let~$e : (Q, \preceq_Q)\epic(R,\preceq_R)$
            \marginnote{\Hyp}

          \step[\imps] $e$ epic in $\mathbf{Set}$
            \marginnote{\Lemma-\ref{lemma:epic-poset-set}}

          \step $P$ epic-preserving in $\mathbf{Set}$
            \marginnote{\Thm-\ref{prop:set-epic-preserving}}

          \step[\imps]
            $\forall f : P \to R\ldotp~\exists h : P \to Q\ldotp~e\compose h = f$
            in $\mathbf{Set}^{\text~(1)}$
            \marginnote{\Def-epic-preserving}

          \step
            \begin{itemize}
              \subp{\dagger}
                \Let~$f : (P,\preceq_P)\to (R,\preceq_R)$
                \marginnote{\Hyp}

              \step[\imps]
                $e\compose h = f$ in $\mathbf{Set}$
                \marginnote{$\exists$-\Elim-($\forall$-\Elim-(1));~$h : P\to Q$}

              \step
                $h$ monotone
                \marginnote{\Lemma-\ref{lemma:discrete-poset-monotone}-($h : P\to Q$)}

              \step[\imps]
                $e\compose h = f$ in $\mathbf{Pos}$

              \step[\imps]
                $\exists h : (P,\preceq_P)\to(Q,\preceq_Q)\ldotp~e\compose h = f$
                \marginnote{$\exists$-\Intro}
            \end{itemize}
        \end{itemize}
        \step[\imps]
          \marginnote[3em]{$\forall$-\Intro-($\forall$-\Intro-$\dagger$)}
          \begin{flalign*}
            & \forall e : (Q,\preceq_Q)\epic(R,\preceq_R)\ldotp &&\\
            & \forall f : (P,\preceq_P)\to(R,\preceq_R)\ldotp &&\\
            & \exists h : (P,\preceq_P)\to(Q,\preceq_Q)\ldotp~e\compose h = f &&
          \end{flalign*}

        \step[\iffs]
          $(P,\preceq_P)$ epic-preserving.
          \qedhere
          \marginnote{\Def-epic-preserving}
    \end{itemize}
  \end{proof}
\end{prop}

\subsection{Part (b.v)}\label{sec:q-1-b-v}

In order to construct an example of a non-epic-preserving poset it suffices to define:

\begin{enumerate}[(i)]
  \item $(Q,\preceq_Q)\in\mathbf{Pos}$
  \item $(R,\preceq_R)\in\mathbf{Pos}$.
  \item $e : (Q,\preceq_Q)\epic(R,\preceq_R)$, an epic morphism in $\mathbf{Pos}$.
  \item $(P,\preceq_P)\in\mathbf{Pos}$.
  \item $f : (P,\preceq_P)\to(R,\preceq_R)$ in $\mathbf{Pos}$.
\end{enumerate}

Such that there is no way to factor $f$ through $e$.\\
Consider the following definitions:

\marginnote{
  In terms of formulae the diagram is given as:
  \begin{align*}
    (P,\preceq_P) & \triangleq\{p_2\succeq_P p_1\preceq_P p_3\}\\
    (Q,\preceq_Q) & \triangleq\{q_2\preceq_Q q_3\succeq_Q q_1\}\\
    (R,\preceq_R) & \triangleq\{r_1\preceq_R r_2\preceq_R r_3\}\\
  \end{align*}
  \begin{align*}
    \forall 1\leq i\leq 3& \ldotp~e : q_i\mapsto ri\\
    \forall 1\leq i\leq 3& \ldotp~f : p_i\mapsto ri
  \end{align*}
}
\begin{center}
  \begin{tikzcd}[column sep=tiny, row sep=large]
    p_3 \arrow[rrrr, dashed, "f", very near end] & & &
    & r_3 &
    & & q_3 \arrow[lll, dashed, "e"', near end] &\\
    & & p_2 \arrow[rr, dashed, "f"] &
    & r_2 \arrow[u, "\preceq_R" description, sloped] &
    & q_2 \arrow[ll, dashed, "e"'] \arrow[ur, "\preceq_Q" description, sloped] & &\\
    & p_1 \arrow[uul, "\succeq_P" description, sloped] \arrow[ur, "\preceq_P" description, sloped, near start] \arrow[rrr, dashed, "f", near end] & &
    & r_1 \arrow[u, "\preceq_R" description, sloped] &
    & & & q_1 \arrow[llll, dashed, "e"', very near end] \arrow[uul, "\succeq_Q" description, sloped]
  \end{tikzcd}
\end{center}

From the definitions, the following propositions hold:

\begin{align*}
  q_1 \preceq_Q q_3 & \implies r_1 \preceq_R r_3 \\
  q_2 \preceq_Q q_3 & \implies r_2 \preceq_R r_3 \\
  p_1 \preceq_P p_2 & \implies r_1 \preceq_R r_2 \\
  p_1 \preceq_P p_3 & \implies r_1 \preceq_R r_3 \\
\end{align*}

Which in turn imply the monotonicity of $e$ and $f$. Furthermore, we have (again by the definition) that $e$ is bijective, and therefore surjective, which in turn implies that $e$ is epic. It suffices, then, to show that there is no way to factor $f$ through $e$.

\begin{prop}
  $\forall h : (P,\preceq_P)\to(Q,\preceq_Q)\ldotp~e\compose h \neq f$. We proceed with this proof by cases.

  \begin{proof}[Case 1]
    Suppose $h(p_1) = q_3$
    \begin{itemize}
      \step[\imps] $\forall 1\leq i\leq 3\ldotp~h(p_i) = q_3$
        \marginnote{\textbf{montone}-$h$}
      \step[\imps] $\forall 1\leq i\leq 3\ldotp~(e\compose h)(p_i) = r_3$
        \marginnote{\Def-$e$}
      \step[\imps] $e\compose h\neq f$
        \marginnote{$f(p_1) = r_1$}
        \qedhere
    \end{itemize}
  \end{proof}

  \begin{proof}[Case 2]
    Suppose $h(p_1) = q_1$
    \begin{itemize}
      \step[\imps] $h(p_2) = q_3 = h(p_3)$
        \marginnote{\textbf{montone}-$h$}
      \step[\imps] $(e\compose h)(p_2) = r_3 \neq r_2 = f(p_2)$
        \marginnote{\Def-$e$;~\Def-$f$}
        \qedhere
    \end{itemize}
  \end{proof}

  \begin{proof}[Case 3]
    Suppose $h(p_1) = q_2$\\
    This case is symmetric to Case 2.\qedhere
  \end{proof}
\end{prop}

\subsection{Part (b.vi)}\label{sec:q-1-b-vi}

\begin{prop}
  Every epic-preserving poset is discrete.

  \begin{proof}
    Let $(P,\preceq_P)\in\mathbf{Pos}$ be epic-preserving.
    \begin{itemize}
      \step
        Define $(P, \preceq_{\widehat{P}})$ to be a discrete poset.

      \step[\imps]
        The inclusion map
        \begin{tikzcd}
          (P,\preceq_{\widehat{P}}) \arrow[r, hook, "i"] & (P, \preceq_P)
        \end{tikzcd}
        is epic in $\mathbf{Pos}$
        \marginnote{$i$ is bijective}

      \step[\imps] The following diagram commutes
        \marginnote[1em]{$(P,\preceq_P)$-\textbf{epic-preserving}}
        \begin{center}
          \begin{tikzcd}[sep=large]
            &  (P, \preceq_{\widehat{P}}) \arrow[d, "i", hook, two heads]
            \\ (P, \preceq_P) \arrow[ur, "h", dashed]
                              \arrow[r, "\id_{(P,\preceq_P)}"'] & (P, \preceq_P)
          \end{tikzcd}
        \end{center}

      \step
        \begin{itemize}
          \subp{\star}
            \Let~$x,y\in P\ldotp~x\preceq_P y$
            \marginnote{\Hyp}

          \step
            \Ass~$x\neq y^{\text~(1)}$
            \marginnote{\Hyp}

          \step[\imps] $h(x) \preceq_{\widehat{P}} h(y)$
            \marginnote{\textbf{monotone}-$h$}

          \step[\imps]
            $h(x) = h(y)$
            \marginnote{\textbf{discrete}-$(P,\preceq_{\widehat{P}})$}

          \step[\imps]
            $(i\compose h)(x) = (i\compose h)(y)$

          \step[\imps]
            $x = y$
            \marginnote{$i\compose h = \id_{(P,\preceq_P)}$}
        \end{itemize}
        \step[\contras] Such an $x,y\in P$ cannot exist
          \marginnote{\Contra-(1)}

        \step[\imps] $(P, \preceq_P)$ discrete\qedhere
    \end{itemize}
  \end{proof}

\end{prop}

\subsection{Part (b.vii)}\label{sec:q-1-b-vii}

Before embarking on this proof, it is useful to define some notation, for $V\in\mathbf{FDVect}_K$.

\newcommand{\Dim}[1]{\mathcal{D}_{#1}}
\begin{definition}[Dimension of $V$]
  Denoted $\Dim{V}$. In the case of any such $V$ we also know that $\Dim{V}$ is finite.
\end{definition}

\newcommand{\Basis}[1]{\mathcal{B}_{#1}}
\begin{definition}[Basis of $V$]
  Denoted $\Basis{V} =\{v_1,\ldots,v_{\Dim{V}}\}$. A minimal orthonormal basis (ONB) of $V$.
\end{definition}

Additionally, observe that a linear map $f : V \to W$ is uniquely defined by its action on a minimal ONB. We will use this fact both when defining linear maps, and proving equivalence between maps.\\[1em]

Now, similar to the proof of Proposition~\ref{prop:set-epic-preserving} (That every object in $\mathbf{Set}$ is epic-preserving), we proceed by proving a property similar to surjectivity in the context of linear maps.

\begin{lemma}\label{lemma:linear-map-epic-span}
  $e : V\epic W \text{ epic} \implies e(\Basis{V}) \text{ spans } W$\\[1em]
  It suffices to prove that $\Basis{W} \subseteq \operatorname{span}(e(\Basis{V}))$.

  \begin{proof}[Proof by Contradiction]
    Let $e : V \epic W$ in $\mathbf{FDVect}_K$
    \begin{itemize}
      \step
        Suppose ${\exists w_i\in\Basis{W}\setminus\operatorname{span}(e(\Basis{V}))}$.
        \marginnote{\Hyp}

      \step
        Then we may show that such a basis vector $w_i$ must be orthogonal to all vectors in $e(\Basis{V})$:

      \step
        \begin{itemize}
          \subp{\star}
            $\nexists \boldsymbol{\lambda}_{e(\Basis{V})}\ldotp~\sum_{y\in e(\Basis{V})} \lambda_y\cdot y = w_i$
            \marginnote{\Def-span;~$\boldsymbol{\lambda}_X$ a sequence indexed by $X$.}

          \step[\iffs]
            $\nexists \boldsymbol{\lambda}_{e(\Basis{V})}\ldotp~\sum_{y\in e(\Basis{V})} \lambda_y\cdot\sum_{j=1}^{\Dim{W}}\mu_{yj}\cdot w_j = w_i$
            \marginnote{$\Basis{W}$ spans $W$;~$\forall y, j\ldotp~\mu_{yj}\in K$}

          \step[\iffs]
            $\nexists \boldsymbol{\lambda}_{e(\Basis{V})}\ldotp~\sum_{y\in e(\Basis{V})} \lambda_y\cdot\mu_{yi}\cdot w_i = w_i$
            \marginnote{$w_i\cdot w_j = \delta_{ij}$}

          \step[\iffs]
            $\nexists \boldsymbol{\lambda}_{e(\Basis{V})}\ldotp~\sum_{y\in e(\Basis{V})} \lambda_y\cdot\mu_{yi} = 1$

          \step[\iffs]
            $\forall y \in e(\Basis{V})\ldotp~\mu_{yi} = 0$
        \end{itemize}


      \step
        Define $g, h : W\to K$ by their action on basis vectors
        \begin{align*}
          g & : w_j \mapsto
          \begin{cases}
            1 & \text{if }w_j \neq w_i\\
            0 & \text{otherwise}
          \end{cases}\\
          h & : w_j \mapsto 1
        \end{align*}

      \step
        \begin{itemize}
          \subp{\dagger}
            So that $g(w_i) = 0 \neq 1 = h(w_i)$
          \step[\imps] $g \neq h$
        \end{itemize}

      \step
        \begin{itemize}
          \subp{\ddagger}
            \Let~$v\in\Basis{V}$
            \marginnote{\Hyp}

          \step $(g\compose e)(v)$

          \step[=] $g(e(v))$
            \marginnote{\Def-$\compose$}

          \step[=]
            $g(\sum_j\mu_jw_j)$
            \marginnote{$\Basis{W}$ spans $W$}

          \step[=]
            $\sum_j\mu_jg(w_j)$
            \marginnote{\textbf{linear}-$g$}

          \step[=]
            $\sum_{j\neq i}\mu_jg(w_j) + \mu_i\cdot g(w_i)$

          \step[=]
            $\sum_{j\neq i}\mu_jg(w_j) + 0\cdot g(w_i)$
            \marginnote{$\forall$-\Elim-$\star$}

          \step[=]
            $\sum_{j\neq i}\mu_jh(w_j) + 0\cdot h(w_i)$
            \marginnote{\Def-$g$;~\Def-$h$}

          \step[=]
            $h(e(v))$
        \end{itemize}

        \step[\imps]
          $\forall v\in\Basis{V}\ldotp~g(e(v))=h(e(v))$
          \marginnote{$\forall$-\Intro}

        \step[\imps]
          $g\compose e = h\compose e$
          \marginnote{\Def-$=$}

        \step[\imps]
          $g = h$
          \marginnote{\Epic-$e$}

        \step[\contras]
          $w_i$ cannot exist
          \marginnote{\Contra-$\dagger$}
          \qedhere
    \end{itemize}
  \end{proof}
\end{lemma}

As a corollary to this lemma, we see that any basis vector $w_i\in\Basis{W}$ can be written as a linear combination of vectors in $e(\Basis{V})$.

\begin{prop}
  All objects (finite-dimensional vector spaces over the field $K$) in $\mathbf{FDVect}_K$ are epic-preserving.

  \begin{proof}
    Let $U\in\mathbf{FDVect}_K$
    \begin{itemize}
      \step
        \begin{itemize}
          \subp{\star}
            \Let~$e : V\epic W$
            \marginnote{\Hyp}

          \step \Let~$f : U \to W$
            \marginnote{\Hyp}

          \item[(i)]
            Observe that for any $w_i\in\Basis{W}$
            \marginnote[1em]{\Lemma-\ref{lemma:linear-map-epic-span}}
            \begin{align*}
              w_i = \sum_{v\in\Basis{V}}\mu_{iv}\cdot e(v)
              \mathnote{$\forall v\in\Basis{V}, 1\leq i\leq\Dim{W}\ldotp~\mu_{iv}\in K$}
            \end{align*}

          \item[(ii)]
            And similarly, for any $u_i\in\Basis{U}$
            \begin{align*}
              f(u_i) = \sum_{k=1}^{\Dim{W}} \lambda_{ik}w_k
              \mathnote{$\Basis{W}$ spans $W$}
            \end{align*}

          \item[(iii)]
            Then consider $h : U \to V$ defined by its action on basis vectors
            \begin{align*}
              h : u_i \mapsto \sum_{k=1}^{\Dim{W}}\sum_{v\in\Basis{V}}\lambda_{ik}\mu_{kv}v
            \end{align*}

          \step
            \begin{itemize}
              \subp{\dagger}
                \Let~$u_i\in\Basis{U}$
                \marginnote{\Hyp}

              \step $f(u_i)$

              \step[=] $\displaystyle\sum\limits_{k=1}^{\Dim{W}}\lambda_{ik}w_k$
                \marginnote{\textbf{observation}-(ii)}

              \step[=] $\displaystyle\sum\limits_{k=1}^{\Dim{W}}\lambda_{ik}\displaystyle\sum\limits_{v\in\Basis{V}}\mu_{kv}e(v).$
                \marginnote{\textbf{observation}-(i)}

              \step[=] $e(\displaystyle\sum\limits_{k=1}^{\Dim{W}}\displaystyle\sum\limits_{v\in\Basis{V}}\lambda_{ik}\mu_{kv}v).$
                \marginnote{\textbf{linear}-$e$}

              \step[=] $e(h(u_i))$
                \marginnote{\Def-$h~\equiv~$(iii)}
            \end{itemize}
            \step[\imps]
              $\forall u_i\in\Basis{U}\ldotp~f(u_i) = e(h(u_i))$
              \marginnote{$\forall$-\Intro}

            \step[\iffs]
              $f = e\compose h$
              \marginnote{\Def-$=$}

            \step[\imps]
              $\exists h : U \to V\ldotp~f = e\compose h$
              \marginnote{$\exists$-\Intro}
        \end{itemize}

        \step[\imps]
          $\forall e : V\epic W, f : U \to W\ldotp~\exists h : U\to V\ldotp~e\compose h = f$
          \marginnote{$\forall$-\Intro}

        \step[\imps]
          $U$ preserves epics
          \marginnote{\Def-epic-preserving}
          \qedhere
    \end{itemize}
  \end{proof}
\end{prop}
