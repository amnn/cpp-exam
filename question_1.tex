\subsection{Part (a.i)}\label{sec:q-1-a-i}

First, we must define a notation for the objects in $\mathbf{Set}$ that our functors will map objects to. The image of both functors is restricted to sets of this form.

\begin{definition}[Hom-set]\label{def:hom-set}
  $\hom{C}{A}{B}\in\mathbf{Set}$. The set of all morphisms from object $A$ to object $B$ in category $\cat{C}$.
\end{definition}

Then we define the action of the functor on objects and morphisms. The covariant Hom-functor maps objects to the set of morphisms \textit{to} that object from $A$, and maps morphisms to functions between Hom-sets (given in the definition).

\begin{definition}[Covariant Hom-functor]\label{def:co-hom}
  $\hom{C}{A}{-} : \cat{C} \to \mathbf{Set}$
  \marginnote{Let $\cat{C}$ be a category, $A\in\cat{C}$}
  \begin{align*}
    \hom{C}{A}{-} & : B \mapsto \hom{C}{A}{B}\\
    \hom{C}{A}{-} & : (f : B \to C) \mapsto (g \mapsto f \compose g)
  \end{align*}
\end{definition}

\begin{marginfigure}[10em]
  \begin{tikzcd}[sep=large]
    & \hom{C}{A}{D} \\
      \hom{C}{A}{B} \arrow[ru,"\hom{C}{A}{g\compose h}"]
                    \arrow[r, "\hom{C}{A}{h}"']
    & \hom{C}{A}{C} \arrow[u, "\hom{C}{A}{g}"']
  \end{tikzcd}
  \caption{Preservation of composition in the image of $\hom{C}{A}{-}$.}\label{fig:co-functor}
\end{marginfigure}

\begin{prop}
  $\hom{C}{A}{-}$ is a functor.

  It suffices to prove that:
  \begin{enumerate}[(i)]
    \item $\hom{C}{A}{-}$ preserves identity morphisms. That is to say, for any $B\in\cat{C}$ it follows that $\hom{C}{A}{\id_B} = \id_{\hom{C}{A}{B}}$
    \item $\hom{C}{A}{-}$ preserves composition. That is to say, for any $B,C,D\in\cat{C}$ with morphisms $g : C \to D$
      and $h : B \to C$, Figure~\ref{fig:co-functor} commutes.
  \end{enumerate}
  \begin{proof}[Proof (i)]
    Let $B\in\cat{C}$
    \begin{itemize}
      \item[$\star$]
        \begin{itemize}
            \item[\phs] \Let~$g : A \to B$
              \marginnote{\Hyp}
            \item[\phs]
              $\hom{C}{A}{\id_B}(g)$

            \item[\eqs]
              $\id_B\compose g$
              \marginnote{\Def-\ref{def:co-hom}}

            \item[\eqs]
              $g$
              \marginnote{\Def-$\id$}

            \item[\eqs]
              $\id_{\hom{C}{A}{B}}(g)$
              \marginnote{\Def-$\id$}
        \end{itemize}

      \item[\imps]
        $\forall g : A \to B\ldotp~\hom{C}{A}{B}(g) = \id_{\hom{C}{A}{B}}(g)$
        \marginnote{$\forall$-\Intro-$\star$}

      \item[\iffs]
        $\hom{C}{A}{B} = \id_{\hom{C}{A}{B}}$
        \marginnote{\Def-=}
        \qedhere
    \end{itemize}
  \end{proof}

  \begin{proof}[Proof (ii)]
    Let $B, C, D\in\cat{C}$ with ${g : C\to D}$, ${h : B\to C}$.
    \begin{itemize}
      \item[$\star$]
        \begin{itemize}
          \item[\phs] \Let~$i : A\to B$
            \marginnote{\Hyp}

          \item[\phs]
            $(\hom{C}{A}{g}\compose\hom{C}{A}{h})(i)$

          \item[\eqs] $\hom{C}{A}{g}(\hom{C}{A}{h}(i))$
            \marginnote{\Def-$\compose$}

          \item[\eqs] $\hom{C}{A}{g}(h\compose i)$
            \marginnote{\Def-\ref{def:co-hom}}

          \item[\eqs] $g\compose (h\compose i)$
            \marginnote{\Def-\ref{def:co-hom}}

          \item[\eqs] $(g\compose h)\compose i$
            \marginnote{\Assoc-$\compose$}

          \item[\eqs] $\hom{C}{A}{g\compose h}(i)$
            \marginnote{\Def-\ref{def:co-hom}}
        \end{itemize}

      \item[\imps]
        $\forall i : A\to B\ldotp~(\hom{C}{A}{g}\compose\hom{C}{A}{h})(i) = \hom{C}{A}{g\compose h}(i)$
        \marginnote{$\forall$-\Intro-$\star$}

      \item[\iffs]
        $\hom{C}{A}{g}\compose\hom{C}{A}{h} = \hom{C}{A}{g\compose h}$
        \marginnote{\Def-$=$}
        \qedhere
    \end{itemize}
  \end{proof}
\end{prop}

Similarly the contravariant Hom-functor maps objects to the set of morphisms \textit{from} that object to A, and maps morphisms to functions between Hom-sets in a corresponding (but not equivalent) way (see below).

\begin{definition}[Contravariant Hom-functor]\label{def:contra-hom}
  $\hom{C}{-}{A} : \opcat{C} \to \mathbf{Set}$
  \marginnote{Let $\cat{C}$ be a category, $A\in\cat{C}$}
  \begin{align*}
    \hom{C}{-}{A} & : B \mapsto \hom{C}{B}{A}\\
    \hom{C}{-}{A} & : (f : B \to C) \mapsto (g \mapsto g \compose f)
  \end{align*}
\end{definition}

\begin{marginfigure}[6em]
  \begin{tikzcd}[sep=large]
    & \hom{C}{D}{A} \\
      \hom{C}{B}{A} \arrow[ru,"\hom{C}{h\compose g}{A}"]
                    \arrow[r, "\hom{C}{h}{A}"']
    & \hom{C}{C}{A} \arrow[u, "\hom{C}{g}{A}"']
  \end{tikzcd}
  \caption{Preservation of composition in the image of $\hom{C}{-}{A}$. Note that the composition $h\compose g$ has been reversed by the definition of composition in the opposite category.}\label{fig:contra-functor}
\end{marginfigure}

\begin{prop}
  $\hom{C}{-}{A}$ is a functor.

  It suffices to prove that:
  \begin{enumerate}[(i)]
    \item $\hom{C}{-}{A}$ preserves identity morphisms. That is to say, for any $B\in\cat{C}$ it follows that $\hom{C}{\id_B}{A} = \id_{\hom{C}{B}{A}}$
    \item $\hom{C}{-}{A}$ preserves composition. That is to say, for any $B,C,D\in\cat{C}$ with morphisms $g : C \to D$
      and $h : B \to C$, Figure~\ref{fig:contra-functor} commutes.
  \end{enumerate}

  \begin{proof}[Proof (i)]
    Let $B\in\cat{C}$
    \begin{itemize}
      \item[$\star$]
        \begin{itemize}
            \item[\phs] \Let~$g : B \to A$
              \marginnote{\Hyp}
            \item[\phs]
              $\hom{C}{\id_B}{A}(g)$

            \item[\eqs]
              $g\compose\id_B$
              \marginnote{\Def-\ref{def:contra-hom}}

            \item[\eqs]
              $g$
              \marginnote{\Def-$\id$}

            \item[\eqs]
              $\id_{\hom{C}{B}{A}}(g)$
              \marginnote{\Def-$\id$}
        \end{itemize}

      \item[\imps]
        $\forall g : B \to A\ldotp~\hom{C}{\id_B}{A}(g) = \id_{\hom{C}{B}{A}}(g)$
        \marginnote{$\forall$-\Intro-$\star$}

      \item[\iffs]
        $\hom{C}{B}{A} = \id_{\hom{C}{B}{A}}$
        \marginnote{\Def-=}
        \qedhere
    \end{itemize}
  \end{proof}

  \begin{proof}[Proof (ii)]
    Let $B, C, D\in\cat{C}$ with ${g : C\to D}$, ${h : B\to C}$.
    \begin{itemize}
      \item[$\star$]
        \begin{itemize}
          \item[\phs] \Let~$i : B\to A$
            \marginnote{\Hyp}

          \item[\phs]
            $(\hom{C}{g}{A}\compose\hom{C}{h}{A})(i)$

          \item[\eqs] $\hom{C}{g}{A}(\hom{C}{h}{A}(i))$
            \marginnote{\Def-$\compose$}

          \item[\eqs] $\hom{C}{g}{A}(i\compose h)$
            \marginnote{\Def-\ref{def:contra-hom}}

          \item[\eqs] $(i\compose h)\compose g$
            \marginnote{\Def-\ref{def:contra-hom}}

          \item[\eqs] $i\compose (h\compose g)$
            \marginnote{\Assoc-$\compose$}

          \item[\eqs] $\hom{C}{h\compose g}{A}(i)$
            \marginnote{\Def-\ref{def:contra-hom}}
        \end{itemize}

      \item[\imps]
        $\forall i : B\to A\ldotp~(\hom{C}{g}{A}\compose\hom{C}{h}{A})(i) = \hom{C}{h\compose g}{A}(i)$
        \marginnote{$\forall$-\Intro-$\star$}

      \item[\iffs]
        $\hom{C}{g}{A}\compose\hom{C}{h}{A} = \hom{C}{h\compose g}{A}$
        \marginnote{\Def-$=$}
        \qedhere
    \end{itemize}
  \end{proof}
\end{prop}

\subsection{Part (a.ii)}\label{sec:q-1-a-ii}

\begin{prop}
  $f$ epic $\iff$ $\hom{C}{f}{C}$ injective for any $C\in\cat{C}$
  \marginnote{Let $f : A \to B$ be a morphism in category $\cat{C}$.}

  \begin{proof}[Proof $\Rightarrow$]
    Let $C\in\cat{C}$.
    \begin{itemize}
      \item[$\star$]
        \begin{itemize}
          \item[\phs] \Let~$g, h : B\to C$
            \marginnote{\Hyp}

          \item[$\dagger$]
            \begin{itemize}
            \item[\phs] \Ass~$\hom{C}{f}{C}(g) = \hom{C}{f}{C}(h)$
              \marginnote{\Hyp}

            \item[\iffs] $g\compose f = h\compose f$
              \marginnote{\Def-\ref{def:contra-hom}}

            \item[\imps] $g = h$
              \marginnote{\Hyp-($f$~\Epic)}
            \end{itemize}

          \item[\imps] $\hom{C}{f}{C}(g) = \hom{C}{f}{C}(h) \implies g = h$
            \marginnote{\imps-\Intro-$\dagger$}
        \end{itemize}
      \item[\imps] $\forall g, h : B \to C\ldotp~\hom{C}{f}{C}(g) = \hom{C}{f}{C}(h) \implies g = h$
        \marginnote{$\forall$-\Intro-$\star$}

      \item[\iffs] $\hom{C}{f}{C}$ injective
        \marginnote{\Def-injective}
        \qedhere
    \end{itemize}
  \end{proof}

  \begin{proof}[Proof $\Leftarrow$]
    Let $C\in\cat{C}$.
    \begin{itemize}
      \item[$\star$]
        \begin{itemize}
          \item[\phs] \Let~$g, h : B\to C$
            \marginnote{\Hyp}

          \item[$\dagger$]
            \begin{itemize}
            \item[\phs] \Ass~$g\compose f = h\compose f$
              \marginnote{\Hyp}

            \item[\iffs] $\hom{C}{f}{C}(g) = \hom{C}{f}{C}(h)$
              \marginnote{\Def-\ref{def:contra-hom}}

            \item[\imps] $g = h$
              \marginnote{\Hyp-($\hom{C}{f}{C}$~\textbf{injective})}
            \end{itemize}

          \item[\imps] $g\compose f = h\compose f \implies g = h$
            \marginnote{\imps-\Intro-$\dagger$}
        \end{itemize}
      \item[\imps] $\forall g, h : B \to C\ldotp~g\compose f = h\compose f \implies g = h$
        \marginnote{$\forall$-\Intro-$\star$}

      \item[\iffs] $f$ epic
        \marginnote{\Def-epic}
        \qedhere
    \end{itemize}
  \end{proof}
\end{prop}

\subsection{Part (a.iii)}\label{sec:q-1-a-iii}
By dualising the result in part (a.ii), we get a characterisation of monics as follows: $f : B\to C$ in $\cat{C}$ is monic if and only if $\hom{C}{A}{f}$ is injective for any $A\in\cat{C}$. The difference being that now the characterisation uses the \textit{covariant} Hom-functor where earlier, when describing epics, it was in terms of the \textit{contravariant} Hom-functor.

\begin{prop}
  $f$ monic $\iff$ $\hom{C}{A}{f}$ injective for any $A\in\cat{C}$
  \marginnote{Let $f : B \to C$ be a morphism in category $\cat{C}$.}

  \begin{proof}[Proof $\Rightarrow$]
    Let $A\in\cat{C}$.
    \begin{itemize}
      \item[$\star$]
        \begin{itemize}
          \item[\phs] \Let~$g, h : A\to B$
            \marginnote{\Hyp}

          \item[$\dagger$]
            \begin{itemize}
            \item[\phs] \Ass~$\hom{C}{A}{f}(g) = \hom{C}{A}{f}(h)$
              \marginnote{\Hyp}

            \item[\iffs] $f\compose g = f\compose h$
              \marginnote{\Def-\ref{def:co-hom}}

            \item[\imps] $g = h$
              \marginnote{\Hyp-($f$~\Monic)}
            \end{itemize}

          \item[\imps] $\hom{C}{A}{f}(g) = \hom{C}{A}{f}(h) \implies g = h$
            \marginnote{\imps-\Intro-$\dagger$}
        \end{itemize}
      \item[\imps] $\forall g, h : A \to B\ldotp~\hom{C}{A}{f}(g) = \hom{C}{A}{f}(h) \implies g = h$
        \marginnote{$\forall$-\Intro-$\star$}

      \item[\iffs] $\hom{C}{A}{f}$ injective
        \marginnote{\Def-injective}
        \qedhere
    \end{itemize}
  \end{proof}

  \begin{proof}[Proof $\Leftarrow$]
    Let $A\in\cat{C}$.
    \begin{itemize}
      \item[$\star$]
        \begin{itemize}
          \item[\phs] \Let~$g, h : A\to B$
            \marginnote{\Hyp}

          \item[$\dagger$]
            \begin{itemize}
            \item[\phs] \Ass~$f\compose g = f\compose h$
              \marginnote{\Hyp}

            \item[\iffs] $\hom{C}{A}{f}(g) = \hom{C}{A}{f}(h)$
              \marginnote{\Def-\ref{def:co-hom}}

            \item[\imps] $g = h$
              \marginnote{\Hyp-($\hom{C}{A}{f}$~\textbf{injective})}
            \end{itemize}

          \item[\imps] $f\compose g = f\compose h \implies g = h$
            \marginnote{\imps-\Intro-$\dagger$}
        \end{itemize}
      \item[\imps] $\forall g, h : A \to B\ldotp~f\compose g = f\compose h \implies g = h$
        \marginnote{$\forall$-\Intro-$\star$}

      \item[\iffs] $f$ monic
        \marginnote{\Def-monic}
        \qedhere
    \end{itemize}
  \end{proof}
\end{prop}

\subsection{Part (b.i)}\label{sec:q-1-b-i}

In order to the prove the main result in this question, it is useful to construct a lemma:

\begin{lemma}\label{lemma:epic-preserve-1}
  For any epic $e : B \epic C$:
  \marginnote{Let $\cat{C}$ be a category and $E\in\cat{C}$.}

  $(\forall f : E\to C\ldotp\exists h : E\to B\ldotp~e\compose h = f)$
  $\iff$ $\hom{C}{E}{e}$ is epic.

  \begin{proof}
    Let epic $e : B \epic C$ in $\cat{C}$.
    \begin{itemize}
      \item[\phs] $\forall f : E\to C\ldotp\exists h : E\to B\ldotp~e\compose h = f$
        \marginnote{\Hyp}
      \item[\iffs]

    $\forall f \in\hom{C}{E}{C}\ldotp\exists h\in\hom{C}{E}{B}\ldotp~\hom{C}{E}{e}(h) = f$
        \marginnote{\Def~\ref{def:hom-set}}

      \item[\iffs]
        $\hom{C}{E}{e}$ surjective.
        \marginnote{\Def-surjective}

      \item[\iffs]
        $\hom{C}{E}{e}$ epic
        \qedhere
        \marginnote{\textbf{sheet} 1, \textbf{q}1}

    \end{itemize}
  \end{proof}
\end{lemma}

Our intended result then follows from the previous lemma, and a general result from first-order logic:

\begin{lemma}\label{lemma:epic-preserve-2}
  If, given sets $X, P, Q$, for any $x\in X$ we have that ${P(x) \iff Q(x)}$ then $\forall x\in X\ldotp~P(x)$ iff $\forall x\in X\ldotp~Q(x)$.

  \begin{proof}
    Let $X, P, Q$ be sets
    \begin{itemize}
      \item[$\star$]
        \begin{itemize}
          \item[\phs]
            \Ass~$\forall x\in X\ldotp~P(x)$
            \marginnote{\Hyp}

          \item[$\dagger$]
            \begin{itemize}
              \item[\phs]
                \Let~$x\in X$
                \marginnote{\Hyp}

              \item[\phs]
                $P(x)$
                \marginnote{$\forall$-\Elim-$\star$}

              \item[\phs]
                $P(x)\iff Q(x)$
                \marginnote{$\forall$-\Elim-($\forall x\in X\ldotp~P(x)\iff Q(x)$)}

              \item[\imps]
                $Q(x)$
                \marginnote{\iffs-\Elim}
            \end{itemize}

            \item[\imps]
              $\forall x\in X\ldotp~Q(x)$
              \marginnote{$\forall$-\Intro-$\dagger$}
        \end{itemize}

      \item[\imps]
        $(\forall x\in X\ldotp~P(x))\implies(\forall x\in X\ldotp~Q(x))$
        \marginnote{\imps-\Intro-$\star$}
      \item[\phantom{\imps}]
        $(\forall x\in X\ldotp~Q(x))\implies(\forall x\in X\ldotp~P(x))$
        \marginnote{follows from a similar argument}
      \item[\iffs]
        $(\forall x\in X\ldotp~P(x))\iff(\forall x\in X\ldotp~Q(x))$
        \qedhere
        \marginnote{\iffs-\Intro}
    \end{itemize}
  \end{proof}
\end{lemma}

All that remains to be done, is to join the two lemmas together (the conclusion of Lemma~\ref{lemma:epic-preserve-1} satisfies the antecedent of Lemma~\ref{lemma:epic-preserve-2}, for a given $X$, $P$, and $Q$):

\begin{marginfigure}
  \begin{tikzcd}[sep=large]
    & B \arrow[d, two heads, "e"]\\
      E \arrow[ur, dashed, "h"]
        \arrow[r, "f"']
    & C
  \end{tikzcd}
  \caption{Epic preserving object, $E$.}\label{fig:epic-preserving}
\end{marginfigure}

\begin{prop}
  $E$ preserves epics $\iff$ for every epic $e : B \epic C$, and every morphism $f : E \to C$, there exists $h : E \to B$ such that Figure~\ref{fig:epic-preserving} commutes.

  \begin{proof}
    Let $E\in\cat{C}$.
    \begin{itemize}
      \item[\phs]
        $E$ preserves epics
        \marginnote{\Hyp}

      \item[\iffs]
        $\hom{C}{E}{-}$ preserves epics
        \marginnote{\Def-(preserves epics)}

      \item[\iffs]
        $\forall e : B \epic C\ldotp$ $\hom{C}{E}{e}$ is epic.
        \marginnote{\Def-(preserves epics)}

      \item[\iffs]
        $\forall e : B\epic C\ldotp~\forall f : E \to C\ldotp\exists h : E \to B\ldotp e\compose h = f$
        \qedhere
        \marginnote{\Lemma-\ref{lemma:epic-preserve-2}-(\Lemma-\ref{lemma:epic-preserve-1})}
    \end{itemize}
  \end{proof}
\end{prop}

\subsection{Part (b.ii)}\label{sec:q-1-b-ii}
\subsection{Part (b.iii)}\label{sec:q-1-b-iii}
\subsection{Part (b.iv)}\label{sec:q-1-b-iv}
\subsection{Part (b.v)}\label{sec:q-1-b-v}
\subsection{Part (b.vi)}\label{sec:q-1-b-vi}
\subsection{Part (b.vii)}\label{sec:q-1-b-vii}
