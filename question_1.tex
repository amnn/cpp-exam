\subsection{Part (a.i)}\label{sec:q-1-a-i}
\begin{definition}[Hom-set]\label{def:hom-set}
  The set of all morphisms from object $A$ to object $B$ in category $\cat{C}$, denoted $\hom{C}{A}{B}$.
\end{definition}

\begin{definition}[Covariant Hom-functor]\label{def:co-hom}
  $\hom{C}{A}{-} : \cat{C} \to \textbf{Set}$
  \marginnote{Let $\cat{C}$ be a category, $A\in\cat{C}$}
  \begin{align*}
    \hom{C}{A}{-} & : B \mapsto \hom{C}{A}{B}\\
    \hom{C}{A}{-} & : (f : B \to C) \mapsto (g \mapsto f \compose g)
  \end{align*}
\end{definition}

\begin{marginfigure}[10em]
  \begin{tikzcd}[sep=large]
    & \hom{C}{A}{D} \\
      \hom{C}{A}{B} \arrow[ru,"\hom{C}{A}{g\compose h}"]
                    \arrow[r, "\hom{C}{A}{h}"']
    & \hom{C}{A}{C} \arrow[u, "\hom{C}{A}{g}"']
  \end{tikzcd}
  \caption{Preservation of composition in the image of $\hom{C}{A}{-}$.}\label{fig:co-functor}
\end{marginfigure}

\begin{prop}
  $\hom{C}{A}{-}$ is a functor.

  It suffices to prove that:
  \begin{enumerate}[(i)]
    \item $\hom{C}{A}{-}$ preserves identity morphisms. That is to say, for any $B\in\cat{C}$ it follows that $\hom{C}{A}{\id_B} = \id_{\hom{C}{A}{B}}$
    \item $\hom{C}{A}{-}$ preserves composition. That is to say, for any $B,C,D\in\cat{C}$ with morphisms $g : C \to D$
      and $h : B \to C$, Figure~\ref{fig:co-functor} commutes.
  \end{enumerate}
  \begin{proof}[Proof (i)]
    Let $B\in\cat{C}$, $g : A \to B$
    \begin{align*}
      \hom{C}{A}{\id_B}(g) & = \id_B\compose g
      \mathnote{\Def-\ref{def:co-hom}}\\
      & = g
      \mathnote{\Def-$\id$}\\
      & = \id_{\hom{C}{A}{B}}(g)
      \mathnote{\Def-$\id$}\\
      \iff \hom{C}{A}{B} & = \id_{\hom{C}{A}{B}}
      \tag*{\qedhere}
    \end{align*}
  \end{proof}

  \begin{proof}[Proof (ii)]
    Let $B, C, D\in\cat{C}$ with ${g : C\to D}$, ${h : B\to C}$ and ${i : A\to B}$.
    \begin{align*}
      (\hom{C}{A}{g}\compose\hom{C}{A}{h})(i)
      & = \hom{C}{A}{g}(\hom{C}{A}{h}(i))
      \mathnote{\Def-$\compose$}
      \\ & = \hom{C}{A}{h}(h\compose i)
      \mathnote{\Def-\ref{def:co-hom}}
      \\ & = g\compose (h\compose i)
      \mathnote{\Def-\ref{def:co-hom}}
      \\ & = (g\compose h)\compose i
      \mathnote{\Assoc-$\compose$}
      \\ & = \hom{C}{A}{g\compose h}(i)
      \mathnote{\Def-\ref{def:co-hom}}
      \\ \iff \hom{C}{A}{g}\compose\hom{C}{A}{h}
      & = \hom{C}{A}{g\compose h}
      \tag*{\qedhere}
    \end{align*}
  \end{proof}
\end{prop}

\begin{definition}[Contravariant Hom-functor]\label{def:contra-hom}
  $\hom{C}{-}{A} : \opcat{C} \to \textbf{Set}$
  \marginnote{Let $\cat{C}$ be a category, $A\in\cat{C}$}
  \begin{align*}
    \hom{C}{-}{A} & : B \mapsto \hom{C}{B}{A}\\
    \hom{C}{-}{A} & : (f : B \to C) \mapsto (g \mapsto g \compose f)
  \end{align*}
\end{definition}

\subsection{Part (a.ii)}\label{sec:q-1-a-ii}
\subsection{Part (a.iii)}\label{sec:q-1-a-iii}
\subsection{Part (b.i)}\label{sec:q-1-b-i}
\subsection{Part (b.ii)}\label{sec:q-1-b-ii}
\subsection{Part (b.iii)}\label{sec:q-1-b-iii}
\subsection{Part (b.iv)}\label{sec:q-1-b-iv}
\subsection{Part (b.v)}\label{sec:q-1-b-v}
\subsection{Part (b.vi)}\label{sec:q-1-b-vi}
\subsection{Part (b.vii)}\label{sec:q-1-b-vii}
