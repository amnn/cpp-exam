\subsection{Part (a)}\label{sec:q-2-a}

First we define some notation for describing $\lambda$-terms and their properties.

\newcommand{\Vars}{\mathcal{V}}

\begin{align*}
  \Vars & \triangleq \{x_1, x_2,\ldots\}
  \mathnote{An inexhaustible supply of variables}\\
  \Lambda & \triangleq\text{ The set of all untyped }\lambda\text{-terms.}\\
  \FV & : \Lambda \to 2^{\Vars}\\
  \FV & : \rlap{$x$}\phantom{\lambda x\ldotp t} \mapsto \{x\}
  \mathnote{$x\in\Vars$}\\
  \FV & : \rlap{$ut$}\phantom{\lambda x\ldotp t} \mapsto FV(u)\cup FV(t)
  \mathnote{$u,t\in\Lambda$}\\
  \FV & : \lambda x\ldotp t\mapsto FV(t)\setminus\{x\}
  \mathnote{$x\in\Vars,~t\in\Lambda$}
\end{align*}

Then we may define $(-)^*$ inductively over the structure of $\lambda$-terms by:

\begin{align*}
  x^* & \triangleq x\\
  (uv)^* & \triangleq \mathbf{r}u^*v^*\\
  (\lambda x\ldotp t)^* & \triangleq \mathbf{s}(\lambda x\ldotp t^*)
\end{align*}

\begin{prop}
  For any $t\in\Lambda$, with $FV(t) = \{x_1,\ldots,x_n\}$, a type deduction of $x_1 : A,\ldots,x_n : A\vdash t^* : A$ exists.\\

  Proof by Induction over structure of $t$
  \begin{proof}[Case] $t\equiv x$ where $x\in\Vars$
    \begin{prooftree}
      \AXC{}
      \RightLabel{\scriptsize(var)}
      \UIC{$x : A\vdash t^*\equiv x : A$}
    \end{prooftree}\qedhere
  \end{proof}

  \begin{proof}[Case]
    $t\equiv uv$ where $u, v\in\Lambda,~\FV(t) =\FV(u)\cup\FV(v)$
    \begin{flalign*}
      \text{W.l.o.g. }\FV(u) & = \{x_1,\ldots,x_i\} &\\
      \FV(v) & = \{x_{j+1},\ldots,x_n\} &
      \mathnote{$0\leq j\leq i\leq n$}
    \end{flalign*}

    \begin{prooftree}
      \AXC{}
      \RightLabel{\scriptsize(\textbf{r}-intro)}
      \UIC{$\vdash \mathbf{r} : A\to A$}
      \AXC{$\vdots$}
      \RightLabel{\scriptsize(I.H.)}
      \UIC{$x_1 : A,\ldots,x_i : A\vdash u^* : A$}
      \RightLabel{\scriptsize($\to$-elim)}
      \BIC{$x_1 : A,\ldots,x_i : A\vdash\mathbf{r}u^* : A\to A$}
      \AXC{$\vdots$}
      \RightLabel{\scriptsize(I.H.)}
      \UIC{$x_{j+1} : A,\ldots,x_n : A\vdash t^* : A$}
      \RightLabel{\scriptsize($\to$-elim)}
      \BIC{$x_1 : A,\ldots,x_n : A\vdash t^* \equiv \mathbf{r}u^*v^* : A$}
    \end{prooftree}
  \end{proof}

  \begin{proof}[Case]
    $t\equiv \lambda x\ldotp u$ where $x\notin \FV(u),~\FV(t)=\FV(u)=\{x_1,\ldots,x_n\}$
    \begin{prooftree}
      \AXC{}
      \RightLabel{\scriptsize(\textbf{s}-intro)}
      \UIC{$\vdash \mathbf{s} : (A\to A)\to A$}
      \AXC{$\vdots$}
      \RightLabel{\scriptsize(I.H.)}
      \UIC{$x_1 : A,\ldots,x_n : A\vdash u^* : A$}
      \RightLabel{\scriptsize($\to$-intro)}
      \UIC{$x_1 : A,\ldots,x_n : A\vdash\lambda x\ldotp u^* : A\to A$}
      \RightLabel{\scriptsize($\to$-elim)}
      \BIC{$x_1 : A,\ldots,x_n : A\vdash t^* \equiv\mathbf{s}(\lambda x\ldotp u^*) : A$}
    \end{prooftree}
  \end{proof}

  \begin{proof}[Case]
    $t\equiv \lambda x\ldotp u$ where $x\in\FV(u),~\FV(t) = \FV(u)\setminus\{x\}$
    \begin{flalign*}
      \text{W.l.o.g. }x & \equiv x_{n+1} &\\
      \FV(u) & = \{x_1,\ldots,x_n,x_{n+1}\} &
    \end{flalign*}
    \begin{prooftree}
      \AXC{}
      \RightLabel{\scriptsize(\textbf{s}-intro)}
      \UIC{$\vdash \mathbf{s} : (A\to A)\to A$}
      \AXC{$\vdots$}
      \RightLabel{\scriptsize(I.H.)}
      \UIC{$x_1 : A,\ldots,x_n : A\vdash u^* : A$}
      \RightLabel{\scriptsize($\to$-intro)}
      \UIC{$x_1 : A,\ldots,x_n : A,x_{n+1} : A\vdash\lambda x\ldotp u^* : A\to A$}
      \RightLabel{\scriptsize($\to$-elim)}
      \BIC{$x_1 : A,\ldots,x_n : A\vdash t^* \equiv\mathbf{s}(\lambda x\ldotp u^*) : A$}
    \end{prooftree}
  \end{proof}
\end{prop}

\subsection{Part (b)}\label{sec:q-2-b}
\subsection{Part (c)}\label{sec:q-2-c}
\subsection{Part (d)}\label{sec:q-2-d}
\subsection{Part (e)}\label{sec:q-2-e}
\subsection{Part (f)}\label{sec:q-2-f}
\subsection{Part (g)}\label{sec:q-2-g}
\subsection{Part (h)}\label{sec:q-2-h}
\subsection{Part (i)}\label{sec:q-2-i}
\subsection{Part (j)}\label{sec:q-2-j}
\subsection{Part (k)}\label{sec:q-2-k}
\subsection{Part (l)}\label{sec:q-2-l}
